\message{ !name(main.tex)}%%%%%%%%%%%%
%% Please rename this main.tex file and the output PDF to
%% [lastname_firstname_graduationyear]
%% before submission.
%%
%% This .tex file is for use with BibLaTeX. Please use
%% main-bibtex.tex instead if you prefer BibTeX.
%%%%%%%%%%%%

\documentclass[12pt, twoside]{unc_so2_template}
\usepackage[hyphens]{url}
\usepackage{lipsum}
\usepackage{graphicx}

\usepackage{todonotes}

%% Tentative: newtx for better-looking Times


% \usepackage[sfdefault]{noto}
\usepackage[utf8]{inputenc}
\usepackage[T1]{fontenc}
% \usepackage{newtxtext,newtxmath}

% Must use biblatex to produce the Published Contents and Contributions, per-chapter bibliography (if desired), etc.
\usepackage[
    backend=biber,natbib,
    % IMPORTANT: load a style suitable for your discipline
    style=ieee
]{biblatex}

% Name of your .bib file(s)
\addbibresource{example.bib}
\addbibresource{ownpubs.bib}

\newcommand\marktext[1]{%
  \textcolor{cyan}{\llap{\smash{\endleft}}}%
  \xblackout{#1}%
  \textcolor{cyan}{\rlap{\smash{\endright}}}%
}

\begin{document}

\message{ !name(main.tex) !offset(-3) }


% Do remember to remove the square bracket!
\title{Anteproyecto del Trabajo Final de Carrera:\\
  Aplicacion del Algoritmo Backpropagation en un Receptor de Comunicaciones Opticas}
\author{Ivan M. Vignolles}
\director{Dr. Ing. Damian A. Morero}

\degreeaward{Ingeniero en Computacion}                 % Degree to be awarded
\university{Facultad de Ciencias Exactas Fisicas y Naturales\newline
  Universidad Nacional de Cordoba}    % Institution name
\address{Córdoba, República Argentina}                     % Institution address
\unilogo{unc1_a.jpg}                                 % Institution logo
\faclogo{fcefyn.png}                                 % Institution logo
\copyyear{2020}  % Year (of graduation) on diploma
% \defenddate{[Exact Date]}          % Date of defense

% \orcid{[Author ORCID]}

%% IMPORTANT: Select ONE of the rights statement below.
% \rightsstatement{All rights reserved\todo[size=\footnotesize]{Choose one from the choices in the source code!! And delete this \texttt{todo} when you're done that. :-)}}
% \rightsstatement{All rights reserved except where otherwise noted}
% \rightsstatement{Some rights reserved. This thesis is distributed under a [name license, e.g., ``Creative Commons Attribution-NonCommercial-ShareAlike License'']}

%%  If you'd like to remove the Caltech logo from your title page, simply remove the "[logo]" text from the maketitle command
\maketitle[logo]
%\maketitle

\begin{acknowledgments}
  \setcounter{page}{3}
   [Add acknowledgements here. If you do not wish to add any to your thesis, you may simply add a blank titled Acknowledgements page.]
\end{acknowledgments}

\begin{abstract}
   [This abstract must provide a succinct and informative condensation of your work. Candidates are welcome to prepare a lengthier abstract for inclusion in the dissertation, and provide a shorter one in the CaltechTHESIS record.]
\end{abstract}

%% Uncomment the `iknowhattodo' option to dismiss the instruction in the PDF.
% \begin{publishedcontent}%[iknowwhattodo]
% % List your publications and contributions here.
% \nocite{Cahn:etal:2015,Cahn:etal:2016}
% \end{publishedcontent}

\tableofcontents
\listoffigures
\listoftables
\printnomenclature

\mainmatter

\chapter{Introducción}
La creciente demanda de información requiere mayores tasas de baudios y ordenes
de modulación, en el caso de las comunicaciones ópticas esto significa que los
tranceptores son menos tolerantes a distorsiones en la transmisión y por lo
tanto es necesario compensar dichas distorsiones de forma más precisa.

Una forma de lograr una compensación más precisa es a través de filtros
adaptivos, cuyos pesos son adaptados utilizando el método del gradiente
descendente estocástico (o SGD por sus siglas en inglés). Sin embargo en un
esquema de sistemas dinámicos en serie y paralelo como suele ser un receptor, el
cálculo del gradiente del error obtenido con respecto a los pesos de los
diferentes filtros utilizando la regla de la cadena se vuelve inviable. Esto es
por el gran número de sub-expresiones que deben ser evaluadas para calcular cada
gradiente, no obstante, muchas de estas sub-expresiones se repiten por lo que se
puede ahorrar costo computacional evaluando una única vez una sub-expresión y
propagando su resultado~\citep{Goodfellow-et-al-2016}.

El concepto de propagar los resultados se formaliza en el algoritmo de
backpropagation (BP)~\citep{Rumelhart1986} que si bien es planteado en el contexto de
redes neuronales es posible aplicarlo para el calculo de derivadas de cualquier
función derivable.

En este trabajo se propone aplicar el algoritmo de backpropagation a través del
tiempo (BPTT por sus siglas en inglés)~\cite{werbos1990}, el cual es un caso
específico del algoritmo de backpropagation para sistemas dinámicos utilizado
comúnmente en redes neuronales recursivas, en un receptor de comunicaciones
ópticas coherentes para adaptar diferentes filtros que componen el mismo.

\section{Objetivos Generales}
El objetivo general de este proyecto es realizar un estudio detallado de la
arquitectura de un receptor de comunicaciones ópticas coherentes para luego
aplicar el algoritmo BPTT para adaptar múltiples filtros que lo componen.

A diferencia del BPTT en redes neuronales donde comúnmente las funciones que se
deben derivar son funciones reales de variables reales, en un receptor de
comunicaciones ópticas existen más relaciones como funciones complejas de
variables reales y funciones reales de variables complejas por lo que deberá
extenderse el algoritmo para tener en cuenta estos diferentes casos.

Se deberá desarrollar un simulador para corroborar el correcto funcionamiento del
receptor como así también la convergencia de sus filtros, el cual debe ser lo
suficientemente flexible para probar el comportamiento de diferentes
arquitecturas de receptor con diferentes efectos del canal.

\section{Objetivos Particulares}

El desarrollo de este proyecto demandará:

\begin{itemize}
\item Adquirir un amplio conocimiento sobre sistemas de comunicaciones ópticas,
  incluyendo arquitecturas de transmisores y receptores, modulación,
  codificación y modelos del canal de comunicación.

\item Adquirir un amplio conocimiento sobre el algoritmo de backpropagation, su
  implementación en sistemas dinámicos y diferenciación de funciones reales y
  complejas de variables reales y complejas.

\item Adquirir un amplio conocimiento sobre el desarrollo de simuladores de
  comunicaciones y computación científica en general.

\item Diseñar y implementar un simulador de comunicaciones extensible, modular y
  configurable que sea capas de almacenar información generada durante
  la simulación como así también graficar dicha información.
\end{itemize}

\chapter{Metodología de Trabajo}
El proyecto constará de las siguientes etapas
\section{Estudio del Estado del Arte de un Sistemas de Comunicaciones Ópticas}
\section{Implementación de un Simulador Básico del Sistema de Comunicaciones}
\section{Desarrollo Teórico de la Derivada de los Diferentes Bloques del Receptor}
\section{Diseño del Simulador de Comunicaciones Aplicando Backpropagation}
\section{Implementación del Simulador}
\section{Verificación del Funcionamiento del Simulador}
\section{Desarrollo del Informe}

\chapter{Backpropagation en un Receptor}
\section{Receptor como un Grafo Computacional}

\chapter{Diseño del Simulador}

\chapter{Resultados Obtenidos}

\chapter{Conclusión}

% \begin{refsection}
% If you'd like to have separate bibliographies at the end of each chapter, put a \verb|refsection| around the material of each chapter, then cite as usual -- e.g.~\citep{GMP81,Ful83}. Then do a \verb|\printbibliography| just before the \verb|refsection| ends. \index{bibliography!by chapter}

% \printbibliography[heading=subbibliography]
% \end{refsection}


% \chapter{This is the Third Chapter}

% \publishedas{Cahn:etal:2015}

% [You can have chapters that were published as part of your thesis. The text style of the body should be single column, as it was submitted to the publisher, not formatted as the publisher did.]

% \chapter{This is the Fourth Chapter}
% \chapter{This is the Fifth Chapter}
% \chapter{This is the Sixth Chapter}
% \chapter{This is the Seventh Chapter}
% \chapter{This is the Eighth Chapter}

\printbibliography[heading=bibintoc]

% \appendix

% \chapter{Questionnaire}
% \chapter{Consent Form}

\printindex

%% Pocket materials at the VERY END of thesis
% \pocketmaterial
% \extrachapter{Pocket Material: Map of Case Study Solar Systems}


\end{document}

\message{ !name(main.tex) !offset(-214) }
