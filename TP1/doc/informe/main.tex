%%%%%%%%%%%%
%% Please rename this main.tex file and the output PDF to
%% [lastname_firstname_graduationyear]
%% before submission.
%%
%% This .tex file is for use with BibLaTeX. Please use
%% main-bibtex.tex instead if you prefer BibTeX.
%%%%%%%%%%%%

\documentclass[11pt]{unc_so2}
\usepackage[hyphens]{url}
\usepackage{lipsum}
\usepackage{graphicx}

\usepackage{todonotes}

%% Tentative: newtx for better-looking Times


% \usepackage[sfdefault]{noto}
\usepackage[utf8]{inputenc}
\usepackage[T1]{fontenc}
% \usepackage{newtxtext,newtxmath}

% Must use biblatex to produce the Published Contents and Contributions, per-chapter bibliography (if desired), etc.
\usepackage[
    backend=biber,natbib,
    % IMPORTANT: load a style suitable for your discipline
    style=ieee
]{biblatex}

% Name of your .bib file(s)
\addbibresource{example.bib}
% \addbibresource{ownpubs.bib}

\newcommand\marktext[1]{%
  \textcolor{cyan}{\llap{\smash{\endleft}}}%
  \xblackout{#1}%
  \textcolor{cyan}{\rlap{\smash{\endright}}}%
}

\begin{document}

% Do remember to remove the square bracket!
\title{Trabajo Práctico Nro I}
\author{Ivan M. Vignolles}
\unilogo{unc1_a.jpg}                                 % Institution logo
\defenddate{11 de Abril de 2019}

\maketitle[logo]

%% Uncomment the `iknowhattodo' option to dismiss the instruction in the PDF.
% \begin{publishedcontent}%[iknowwhattodo]
% % List your publications and contributions here.
% \nocite{Cahn:etal:2015,Cahn:etal:2016}
% \end{publishedcontent}

\nombrefecha
\tableofcontents*
\clearpage
% \listoffigures
% \listoftables
% \printnomenclature

\mainmatter

\section{Introducción}\label{sec:intr}
Los sockets son una abstracción de comunicación entre procesos (IPC) que, en un
sistema tipo UNIX, se implementan en un descriptor de archivo, sobre el cual se
envía o recibe información, al igual que como se lee o escribe un archivo
\textbf{CITATION NEEDED}. Son una herramienta muy importante, uno de los pilares
de la comunicación entre procesos, y sumamente utilizada en la mayoría de las
aplicaciones de red.

\subsection{Propósito}
El propósito de este trabajo es aplicar los conocimientos adquiridos sobre
sockets y llamadas de sistema en general en una aplicación del tipo cliente-servidor que
emula la comunicación entre una estación terrena, representada por una PC, y un
satélite, que en este caso particular es representado por una Raspberry Pi.

\subsection{Ámbito del Sistema}
Tanto cliente como servidor correrán sobre Linux, el servidor proporcionará un
prompt a partir del cual se podrá interactuar con el cliente y hacer diferentes peticiones.

\subsection{Definiciones, Acrónimos y Abreviaturas}

\subsubsection{TCP}
El protocolo de control de transmisión (TCP por sus siglas en ingles) es uno de
los protocolos de la capa de transmisión más utilizados, permite crear
conexiones entre dispositivos y de esta manera comunicar un flujo de datos~\cite{tcp1}.

Entre las principales características del protocolo TCP se pueden mencionar las
siguientes: permite poner nuevamente los datagramas en orden cuando vienen del
protocolo IP, permite que el monitoreo del flujo de los datos y así evita la
saturación de la red, permite que los datos se formen en segmentos de longitud
variada para entregarlos al protocolo IP, permite multiplexar los datos, es
decir, que la información que viene de diferentes fuentes (por ejemplo,
aplicaciones) en la misma línea pueda circular simultáneamente. Por último, TCP
permite comenzar y finalizar la comunicación amablemente~\cite{tcp2}.

\subsubsection{UDP}
El protocolo de datagramas de usuario (en inglés: User Datagram Protocol o UDP)
es un protocolo del nivel de transporte basado en el intercambio de datagramas.
Permite el envío de datagramas a través de la red sin que se haya establecido
previamente una conexión, ya que el propio datagrama incorpora suficiente
información de direccionamiento en su cabecera. Tampoco tiene confirmación ni
control de flujo, por lo que los paquetes pueden adelantarse unos a otros; y
tampoco se sabe si ha llegado correctamente, ya que no hay confirmación de
entrega o recepción~\cite{udp}.

\subsubsection{RTT}
El tiempo de ida y vuelta (RTT, round-trip time o round-trip delay time) se
aplica en el mundo de las telecomunicaciones y redes informáticas al tiempo que
tarda un paquete de datos enviado desde un emisor en volver a este mismo emisor
habiendo pasado por el receptor de destino~\cite{rtt}.

\subsubsection{Prompt}
Se llama prompt al carácter o conjunto de caracteres que se muestran en una
línea de comandos para indicar que está a la espera de órdenes. Este puede
variar dependiendo del intérprete de comandos y suele ser configurable~\cite{prompt}.

\subsection{Descripción General del Documento}
El documento se organiza de la siguiente forma: la sección~\ref{sec:descr}
brinda una descripción del proyecto, incluyendo requisitos, funciones, y
características. La sección~\ref{sec:espec} profundiza los requisitos del
sistema. En la sección~\ref{sec:design} se provee el diseño y en la
sección~\ref{sec:impl} se muestran los resultados obtenidos.

\section{Descripción General}\label{sec:descr}
\subsection{Perspectiva del Producto}
Tanto la aplicación del cliente como la del servidor serán desarrolladas en
lenguaje C. Con el objetivo de simplificar la implementación, teniendo en cuenta
que los procesos deben ser mono-thread, el servidor soportará únicamente una
conexión a la vez.

Ambos procesos se ejecutaran sobre Linux, parte de la comunicación debe darse
sobre un protocolo orientado a la conexión (TCP), y otra porción sobre protocolo
no orientado a la conexión (UDP). Se debe hacer un uso eficiente de los buffers
de comunicación de forma tal que asegure una comunicación veloz sin tener un
gran impacto en la memoria RAM.

El servidor deberá solicitar una autenticación cada vez que ocurra
una conexión para dar acceso al prompt desde el cual se pueden hacer consultas
al cliente.

\subsection{Funciones del Producto}
\subsubsection{Conexión}
El sistema conecta dos dispositivos a través de internet y permite que interactúen.

\subsubsection{Autenticación}
Cada vez que una conexión ocurra, el servidor pedirá que el usuario se
autentifique utilizando una contraseña antes de permitirle hacer cualquier
petición al cliente.


\subsubsection{Consulta de telemetría}
El servidor puede hacer una petición al cliente para que éste le envíe
información de sus recursos.

\subsubsection{Consulta de escaneo}
Se puede hacer una petición al cliente para recibir un escaneo de la Tierra.

\subsubsection{Actualización de firmware}
El servidor puede enviar una nueva versión del firmware del cliente y dar la
orden de ejecutarlo, a continuación, el cliente reiniciará la conexión corriendo
la versión actualizada.

\subsection{Características de los Usuarios}
Los usuarios deben tener un conocimiento básico de como manejar una consola,
como así también deben conocer los comandos que el sistema provee.

\subsection{Restricciones}
Las principales restricciones del sistema se detallan a continuación:

\begin{itemize}
  \item El servidor solo soporta una conexión a la vez.
  \item La contraseña de autenticación no puede ser modificada.
  \item El nombre del archivo que se pase cuando se de la orden de actualizar el firmware
    debe ser un archivo de firmware válido.
  \item El nombre de la imagen recibida al solicitar un escaneo no puede ser
    modificado, en caso de que un archivo con el mismo nombre ya exista, sera sobrescrito.
  \item El código del cliente y el servidor deben ser escritos en el lenguaje C,
    con el estilo de código descrito por el kernel de linux y debe ser
    corroborado con la utilidad \textit{cppcheck}.
\end{itemize}

\subsection{Suposiciones y Dependencias}
Se asume que tanto el cliente como el servidor corren sobre un sistema operativo
Linux, tienen permisos para escribir en el disco, crear procesos, crear sockets
y realizar conexiones.

Se asume que la dirección IP del servidor que se encuentra en el código de
fuente es la correcta y que los puertos definidos son libres de ser utilizados.

Se asume que el tamaño de la imagen de escaneo es considerablemente menor que la
memoria RAM disponible tanto para el cliente como para el servidor y que existe
espacio suficiente en la memoria secundaria para escribirla.

\section{Requisitos Específicos}\label{sec:espec}
\subsection{Interfaces Externas}
\subsection{Funciones}
\subsection{Requisitos de Rendimiento}
\subsection{Restricciones de Diseño}
\subsection{Atributos del Sistema}
\section{Diseño de solución}\label{sec:design}
\section{Implementación y Resultados}\label{sec:impl}
\section{Conclusiones}\label{sec:concl}
\section{Apéndices}\label{sec:apnd}

\printbibliography[heading=subbibintoc]

\end{document}
