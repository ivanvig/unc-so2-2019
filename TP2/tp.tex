%%%%%%%%%%%%%%%%%%%%%%%%%%%%%%%%%%%%%%%%%
% Uppsala University Assignment Title Page 
% LaTeX Template
% Version 1.0 (27/12/12)
%
% This template has been downloaded from:
% http://www.LaTeXTemplates.com
%
% Original author:
% WikiBooks (http://en.wikibooks.org/wiki/LaTeX/Title_Creation)
% Modified by Elsa Slattegard to fit Uppsala university
% License:
% CC BY-NC-SA 3.0 (http://creativecommons.org/licenses/by-nc-sa/3.0/)

%\title{Title page with logo}
%----------------------------------------------------------------------------------------
%	PACKAGES AND OTHER DOCUMENT CONFIGURATIONS
%----------------------------------------------------------------------------------------

\documentclass[12pt]{article}
\usepackage[english]{babel}
\usepackage[utf8x]{inputenc}
\usepackage{amsmath}
\usepackage{graphicx}
\renewcommand{\figurename}{Figura}
\usepackage{float}
\usepackage[colorinlistoftodos]{todonotes}

\begin{document}

\begin{titlepage}
\newcommand{\HRule}{\rule{\linewidth}{0.5mm}} % Defines a new command for the horizontal lines, change thickness here

\center % Center everything on the page
 
%----------------------------------------------------------------------------------------
%	HEADING SECTIONS
%----------------------------------------------------------------------------------------

\textsc{\LARGE Universidad Nacional de C\'ordoba}\\[1.5cm] % Name of your university/college
\includegraphics[scale=.4]{unc.png}\\[1cm] % Include a department/university logo - this will require the graphicx package
\textsc{\Large Facultad de Ciencias Exactas, F\'isicas y Naturales}\\[0.5cm] % Major heading such as course name
\textsc{\large Sistemas Operativos II}\\[0.5cm] % Minor heading such as course title

%----------------------------------------------------------------------------------------
%	TITLE SECTION
%----------------------------------------------------------------------------------------

\HRule \\[0.4cm]
{ \LARGE \bfseries Trabajo Pr\'actico N° 2: OpenMP}\\[0.4cm] % Title of your document
\HRule \\[1.5cm]
 
%----------------------------------------------------------------------------------------
%	AUTHOR SECTION
%----------------------------------------------------------------------------------------

% \begin{minipage}{0.4\textwidth}
% \begin{flushleft} \large
% \emph{Author:}\\
% Ivan M. \textsc{Vignolles}\\ % Your name
% \end{flushleft}

% \end{minipage}\\[2cm]

% If you don't want a supervisor, uncomment the two lines below and remove the section above
\Large \emph{Alumno:}\\
Ivan M. \textsc{Vignolles}\\[1.5cm] % Your name

%----------------------------------------------------------------------------------------
%	DATE SECTION
%----------------------------------------------------------------------------------------

% {\large \today}\\%[2cm] % Date, change the \today to a set date if you want to be precise
{\large Mayo de 2019}\\[2cm] % Date, change the \today to a set date if you want to be precise

\vfill % Fill the rest of the page with whitespace

\end{titlepage}

\section{Objetivo}
El objetivo del trabajo es implementar la operaci\'on de convoluci\'on discreta
en dos dimensiones y aplicarla junto con un kernel de detecci\'on de bordes a la
imagen que se observa en la Fig.~\ref{mundo}, luego paralelizar dicho algoritmo y analizar el cambio en
los tiempos de ejecuci\'on.

\begin{figure}
\centering
\includegraphics[scale=.5]{planeta}
\caption{Imagen utilizada en el desarrollo del trabajo.}
\label{mundo}
\end{figure}

\end{document}