%%%%%%%%%%%%%%%%%%%%%%%%%%%%%%%%%%%%%%%%%
% Uppsala University Assignment Title Page 
% LaTeX Template
% Version 1.0 (27/12/12)
%
% This template has been downloaded from:
% http://www.LaTeXTemplates.com
%
% Original author:
% WikiBooks (http://en.wikibooks.org/wiki/LaTeX/Title_Creation)
% Modified by Elsa Slattegard to fit Uppsala university
% License:
% CC BY-NC-SA 3.0 (http://creativecommons.org/licenses/by-nc-sa/3.0/)

%\title{Title page with logo}
%----------------------------------------------------------------------------------------
%	PACKAGES AND OTHER DOCUMENT CONFIGURATIONS
%----------------------------------------------------------------------------------------

\documentclass[12pt]{article}
\usepackage[english]{babel}
\usepackage[utf8x]{inputenc}
\usepackage{amsmath}
\usepackage{graphicx}
\renewcommand{\figurename}{Figura}
\usepackage{float}
\usepackage[colorinlistoftodos]{todonotes}

\begin{document}

\begin{titlepage}
\newcommand{\HRule}{\rule{\linewidth}{0.5mm}} % Defines a new command for the horizontal lines, change thickness here

\center % Center everything on the page
 
%----------------------------------------------------------------------------------------
%	HEADING SECTIONS
%----------------------------------------------------------------------------------------

\textsc{\LARGE Universidad Nacional de C\'ordoba}\\[1.5cm] % Name of your university/college
\includegraphics[scale=.4]{unc.png}\\[1cm] % Include a department/university logo - this will require the graphicx package
\textsc{\Large Facultad de Ciencias Exactas, F\'isicas y Naturales}\\[0.5cm] % Major heading such as course name
\textsc{\large Sistemas Operativos II}\\[0.5cm] % Minor heading such as course title

%----------------------------------------------------------------------------------------
%	TITLE SECTION
%----------------------------------------------------------------------------------------

\HRule \\[0.4cm]
{ \LARGE \bfseries Trabajo Pr\'actico N° 2: OpenMP}\\[0.4cm] % Title of your document
\HRule \\[1.5cm]
 
%----------------------------------------------------------------------------------------
%	AUTHOR SECTION
%----------------------------------------------------------------------------------------

% \begin{minipage}{0.4\textwidth}
% \begin{flushleft} \large
% \emph{Author:}\\
% Ivan M. \textsc{Vignolles}\\ % Your name
% \end{flushleft}

% \end{minipage}\\[2cm]

% If you don't want a supervisor, uncomment the two lines below and remove the section above
\Large \emph{Alumno:}\\
Ivan M. \textsc{Vignolles}\\[1.5cm] % Your name

%----------------------------------------------------------------------------------------
%	DATE SECTION
%----------------------------------------------------------------------------------------

% {\large \today}\\%[2cm] % Date, change the \today to a set date if you want to be precise
{\large Mayo de 2019}\\[2cm] % Date, change the \today to a set date if you want to be precise

\vfill % Fill the rest of the page with whitespace

\end{titlepage}

\section{Objetivo}
El objetivo del trabajo es implementar en C la operaci\'on de convoluci\'on discreta
en dos dimensiones y aplicarla junto con un kernel de detecci\'on de bordes a la
imagen que se observa en la Fig.~\ref{mundo}, luego paralelizar dicho algoritmo
utilizando \textit{OpenMP} y analizar el cambio en los tiempos de ejecuci\'on.

\begin{figure}[H]
\centering
\includegraphics[scale=.5]{planeta}
\caption{Imagen utilizada en el desarrollo del trabajo.}\label{mundo}
\end{figure}

\section{Desarrollo}
\subsection{Carga de la imagen a memoria}
Los datos de la imagen se encuentran en un archivo de tipo \textit{nc} por lo
que para poder acceder a ellos es necesario instalar la librer\'ia
\textit{NetCDF4}. La imagen tiene un tamaño de $21696\times21696$ pixels, donde
cada pixel es una variable del tipo \textit{float} de 32 bits cuyo rango
v\'alido de valores es $0 - 4096$ y representa el nivel de brillo.

\subsection{Ejecuci\'on procedural}
En primera instancia se implement\'o la convoluci\'on de forma procedural, esto
es, un solo hilo es el encargado de ejecutar las instrucciones del programa.

Se desarrollaron dos algoritmos diferentes para aplicar la convoluci\'on en 2D,
el primero corresponde a la forma cl\'asica que consta de cuatro bucles anidados
y realiza la acumulaci\'on de todos productos de los coeficientes del kernel con
la secci\'on correspondiente de la imagen antes de desplazarse horizontal o
verticalmente para generar un nuevo pixel de salida. La salida producida por
este m\'etodo se considera la deseada y es utilizada como referencia para
corroborar que las salidas generadas por los algoritmos son correctas.

El segundo algoritmo intenta hacer un uso mas eficiente de los datos ya
disponibles en la memoria cache teniendo en cuenta que la ubicac\'ion de los
datos en memoria tanto de la imagen original como la procesada es \textit{row
  major}. Para ello, toma una fila de la imagen original y realiza 3
convoluciones 1D con cada fila del kernel (considerando un kernel de 3 filas)
produciendo as\'i 3 filas con resultados parciales. Luego se pasa la fila
siguiente de la imagen y se producen otras 3 nuevas filas de resultados
parciales que se acumulan con las anteriores de forma tal que el resultado
obtenido se corresponda con una convoluci\'on 2D.

\subsection{Ejecuci\'on paralela}
Se paralelizaron ambos algoritmos descritos en la secci\'on anterior, para ello
se hizo uso de los pragmas \texttt{parallel}, y \texttt{for} que provee OpenMP.
No fue necesario el uso de m\'etodos de sincronizaci\'on como \texttt{critical}
o \texttt{atomic} pues se considera que no existe alg\'un caso donde pudiera
haber un conflicto entre hilos.

\section{Resultados}
Para obtener los resultados se utilizaron dos computadoras, una local

\begin{table}
% increase table row spacing, adjust to taste
\renewcommand{\arraystretch}{1.3}
% if using array.sty, it might be a good idea to tweak the value of
% \extrarowheight as needed to properly center the text within the cells
\caption{Utilización de recursos con y sin DSP.}
\label{res_table}
\centering
% some packages, such as mdw tools, offer better commands for making tables
% than the plain latex2e tabular which is used here.
\footnotesize
\begin{tabular}{|l|p{5cm}|}
  \hline
Arquitectura                        & x86\_64 \\ \hline
modo(s) de operación de las CPUs    & 32-bit, 64-bit \\ \hline
Orden de los bytes                  & Little Endian \\ \hline
Tamaños de las direcciones          & 39 bits physical, 48 bits virtual \\ \hline
CPU(s)                              & 4 \\ \hline
Lista de la(s) CPU(s) en línea      & 0-3 \\ \hline
Hilo(s) de procesamiento por núcleo & 1 \\ \hline
Núcleo(s) por «socket»              & 4 \\ \hline
«Socket(s)»                          & 1 \\ \hline
Modo(s) NUMA                        & 1 \\ \hline
ID de fabricante                    & GenuineIntel \\ \hline
Familia de CPU                      & 6 \\ \hline
Modelo                              & 94 \\ \hline
Nombre del modelo                   & Intel(R) Core(TM) i5-6500 CPU @ 3.20GHz \\ \hline
Revisión                            & 3 \\ \hline
CPU MHz                             & 800.039 \\ \hline
CPU MHz máx.                        & 3600,0000 \\ \hline
CPU MHz mín.                        & 800,0000 \\ \hline
BogoMIPS                            & 6386.00 \\ \hline
Virtualización                      & VT-x \\ \hline
Caché L1d                           & 32K \\ \hline
Caché L1i                           & 32K \\ \hline
Caché L2                            & 256K \\ \hline
Caché L3                            & 6144K \\ \hline
CPU(s) del nodo NUMA 0              & 0-3 \\ \hline
Indicadores                         & \textbf{fpu vme de pse tsc msr pae mce cx8 apic sep mtrr pge mca cmov pat pse36 clflush dts acpi mmx fxsr sse sse2 ss ht tm pbe syscall nx pdpe1gb rdtscp lm constant_tsc art arch_perfmon pebs bts rep_good nopl xtopology nonstop_tsc cpuid aperfmperf tsc_known_freq pni pclmulqdq dtes64 monitor ds_cpl vmx smx est tm2 ssse3 sdbg fma cx16 xtpr pdcm pcid sse4_1 sse4_2 x2apic movbe popcnt tsc_deadline_timer aes xsave avx f16c rdrand lahf_lm abm 3dnowprefetch cpuid_fault epb invpcid_single pti ssbd ibrs ibpb stibp tpr_shadow vnmi flexpriority ept vpid ept_ad fsgsbase tsc_adjust bmi1 hle avx2 smep bmi2 erms invpcid rtm mpx rdseed adx smap clflushopt intel_pt xsaveopt xsavec xgetbv1 xsaves dtherm ida arat pln pts hwp hwp_notify hwp_act_window hwp_epp flush_l1d}
\end{tabular}           
\end{table}
\end{document}